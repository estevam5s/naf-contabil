\documentclass[12pt,a4paper]{article}
\usepackage[utf8]{inputenc}
\usepackage[portuguese]{babel}
\usepackage{geometry}
\usepackage{fancyhdr}
\usepackage{graphicx}
\usepackage{xcolor}
\usepackage{hyperref}
\usepackage{enumitem}
\usepackage{titlesec}

\geometry{margin=2.5cm}

\pagestyle{fancy}
\fancyhf{}
\fancyhead[L]{NAF - Núcleo de Apoio Fiscal}
\fancyhead[R]{Estácio Florianópolis}
\fancyfoot[C]{\thepage}

\titleformat{\section}{\large\bfseries\color{blue!60!black}}{\thesection}{1em}{}
\titleformat{\subsection}{\normalsize\bfseries\color{blue!40!black}}{\thesubsection}{1em}{}

\begin{document}

\begin{center}
{\LARGE \textbf{GUIA COMPLETO}}\\[0.5cm]
{\Large \textbf{e-SOCIAL DOMÉSTICO}}\\[0.5cm]
{\large Sistema de Escrituração Digital das Obrigações}\\[0.3cm]
{\large Fiscais, Previdenciárias e Trabalhistas}\\[0.3cm]
{\large Núcleo de Apoio Fiscal - NAF}\\
{\large Estácio Florianópolis}\\[1cm]
\end{center}

\section{O que é o e-Social Doméstico?}

O e-Social Doméstico é um sistema criado pelo governo federal para unificar o envio de informações trabalhistas, previdenciárias, tributárias e fiscais dos empregados domésticos, substituindo várias obrigações separadas por uma única ferramenta digital.

\section{Quem é considerado empregado doméstico?}

\subsection{Definição legal}
Empregado doméstico é aquele que presta serviços de forma contínua, subordinada, onerosa e pessoal, no âmbito residencial de pessoa ou família, sem finalidade lucrativa para o empregador.

\subsection{Exemplos de profissionais}
\begin{itemize}
    \item Cozinheira(o)
    \item Faxineira(o)
    \item Lavadeira(o)
    \item Passadeira(o)
    \item Babá
    \item Cuidadora(or) de idosos
    \item Motorista particular
    \item Jardineiro
    \item Caseiro
    \item Governanta
\end{itemize}

\section{Quem deve usar o e-Social Doméstico?}

\subsection{Empregadores obrigatórios}
\begin{itemize}
    \item Pessoa física que possui empregado doméstico
    \item Família que contrata trabalhador doméstico
    \item Condomínio que contrata funcionários para área comum residencial
\end{itemize}

\subsection{Quando é obrigatório}
\begin{itemize}
    \item Trabalho com frequência superior a 2 dias por semana
    \item Vínculo empregatício formal
    \item Independente do valor do salário
    \item Desde o primeiro dia de trabalho
\end{itemize}

\section{Direitos dos empregados domésticos}

\subsection{Direitos garantidos pela Lei Complementar 150/2015}
\begin{itemize}
    \item Salário mínimo ou piso da categoria
    \item Jornada de trabalho de 8h diárias e 44h semanais
    \item Horas extras com adicional de 50\%
    \item Repouso semanal remunerado
    \item Feriados remunerados
    \item Férias de 30 dias com adicional de 1/3
    \item 13º salário
    \item Licença-maternidade de 120 dias
    \item Licença-paternidade
    \item Aviso prévio proporcional
    \item Estabilidade da gestante
    \item FGTS
    \item Seguro-desemprego
    \item Adicional noturno (22h às 5h)
    \item Salário-família (se aplicável)
\end{itemize}

\section{Como cadastrar no e-Social Doméstico}

\subsection{Primeiro acesso}
\begin{enumerate}
    \item Acesse: \url{https://www.esocial.gov.br/portal/}
    \item Clique em "Doméstico"
    \item Selecione "Sou empregador doméstico"
    \item Faça login com CPF e senha
    \item Crie sua conta se não tiver
    \item Preencha seus dados pessoais
    \item Confirme o cadastro
\end{enumerate}

\subsection{Cadastro do empregado}
\begin{enumerate}
    \item Entre no sistema
    \item Clique em "Cadastrar empregado"
    \item Preencha dados pessoais
    \item Informe dados contratuais
    \item Defina salário e função
    \item Inclua dependentes (se houver)
    \item Salve as informações
\end{enumerate}

\section{Funcionalidades do sistema}

\subsection{Cadastros}
\begin{itemize}
    \item Dados do empregador
    \item Dados do empregado
    \item Dependentes
    \item Contratos de trabalho
    \item Alterações contratuais
\end{itemize}

\subsection{Folha de pagamento}
\begin{itemize}
    \item Cálculo automático de salários
    \item Desconto de INSS e IRRF
    \item Cálculo de horas extras
    \item Adicional noturno
    \item DSR sobre horas extras
    \item 13º salário
    \item Férias
\end{itemize}

\subsection{Documentos gerados}
\begin{itemize}
    \item DAE (Documento de Arrecadação do e-Social)
    \item Recibo de pagamento
    \item Comprovante de recolhimento
    \item Relatórios diversos
    \item Cartão de ponto eletrônico
\end{itemize}

\section{Como funciona o DAE}

\subsection{O que é}
O DAE (Documento de Arrecadação do e-Social) unifica o recolhimento de:
\begin{itemize}
    \item INSS do empregado (8\% a 11\%)
    \item INSS patronal (12\%)
    \item FGTS (8\%)
    \item Imposto de Renda na fonte (se aplicável)
    \item Seguro contra acidentes (0,8\%)
\end{itemize}

\subsection{Quando gerar}
\begin{itemize}
    \item Mensalmente, até o dia 7
    \item Para o mês de competência anterior
    \item Mesmo se não houver alterações no salário
    \item Separadamente para cada empregado
\end{itemize}

\subsection{Como pagar}
\begin{itemize}
    \item Internet banking
    \item Caixas eletrônicos
    \item Agências bancárias
    \item Casas lotéricas
    \item Correspondentes bancários
\end{itemize}

\section{Obrigações mensais}

\subsection{Até o dia 7 de cada mês}
\begin{itemize}
    \item Gerar a folha de pagamento
    \item Conferir dados e valores
    \item Emitir o DAE
    \item Efetuar o pagamento
    \item Guardar comprovantes
\end{itemize}

\subsection{Durante o mês}
\begin{itemize}
    \item Registrar ponto (se aplicável)
    \item Controlar horas extras
    \item Anotar faltas e atrasos
    \item Registrar afastamentos
    \item Atualizar dados quando necessário
\end{itemize}

\section{Eventos especiais}

\subsection{Admissão}
\begin{itemize}
    \item Cadastrar empregado antes do início
    \item Informar dados contratuais
    \item Definir salário e função
    \item Gerar primeiro DAE
\end{itemize}

\subsection{Alteração contratual}
\begin{itemize}
    \item Aumento salarial
    \item Mudança de função
    \item Alteração de jornada
    \item Inclusão/exclusão de dependentes
\end{itemize}

\subsection{Afastamento}
\begin{itemize}
    \item Licença-maternidade
    \item Auxílio-doença
    \item Acidente de trabalho
    \item Afastamento sem remuneração
\end{itemize}

\subsection{Desligamento}
\begin{itemize}
    \item Informar data e motivo
    \item Calcular verbas rescisórias
    \item Gerar DAE final
    \item Efetuar pagamentos
\end{itemize}

\section{Cálculo do 13º salário}

\subsection{Primeira parcela}
\begin{itemize}
    \item Paga entre fevereiro e novembro
    \item 50\% do salário vigente
    \item Sem descontos
    \item Opcional para o empregador
\end{itemize}

\subsection{Segunda parcela}
\begin{itemize}
    \item Paga até 20 de dezembro
    \item Valor total menos primeira parcela
    \item Com descontos de INSS e IRRF
    \item Proporcional ao tempo trabalhado
\end{itemize}

\section{Cálculo das férias}

\subsection{Direito}
\begin{itemize}
    \item 30 dias após 12 meses de trabalho
    \item Adicional de 1/3 sobre o salário
    \item Pagamento até 2 dias antes do início
    \item Pode ser dividida em até 3 períodos
\end{itemize}

\subsection{Cálculo}
\begin{itemize}
    \item Salário integral do mês
    \item Mais 1/3 constitucional
    \item Descontos de INSS e IRRF
    \item Proporcional se menor que 12 meses
\end{itemize}

\section{Rescisão de contrato}

\subsection{Verbas rescisórias}
\begin{itemize}
    \item Salário proporcional
    \item Férias vencidas e proporcionais + 1/3
    \item 13º salário proporcional
    \item Aviso prévio (se aplicável)
    \item Multa de 40\% do FGTS (demissão sem justa causa)
\end{itemize}

\subsection{Prazos para pagamento}
\begin{itemize}
    \item Até 10 dias corridos da rescisão
    \item Independente do tipo de demissão
    \item Multa por atraso: 1 salário-mínimo
\end{itemize}

\section{Controle de ponto}

\subsection{Quando obrigatório}
\begin{itemize}
    \item Jornada superior a 6 horas diárias
    \item Pode ser manual ou eletrônico
    \item Assinatura do empregado
    \item Guarda por 5 anos
\end{itemize}

\subsection{Como fazer}
\begin{itemize}
    \item Livro de ponto manual
    \item Aplicativo do e-Social
    \item Planilhas eletrônicas
    \item Sistemas terceirizados
\end{itemize}

\section{Penalidades}

\subsection{Multas por atraso}
\begin{itemize}
    \item Cadastro: R\$ 180,00 por empregado
    \item Folha de pagamento: 2\% sobre contribuições
    \item Mínimo: R\$ 180,00
    \item Máximo: R\$ 1.800,00
\end{itemize}

\subsection{Multas por omissão}
\begin{itemize}
    \item Não cadastrar: R\$ 3.000,00
    \item Informações incorretas: R\$ 600,00 a R\$ 6.000,00
    \item Não pagar contribuições: Juros e multa de mora
\end{itemize}

\section{Dicas importantes}

\begin{itemize}
    \item Mantenha dados sempre atualizados
    \item Guarde todos os comprovantes
    \item Pague o DAE sempre em dia
    \item Faça backup das informações
    \item Consulte regularmente o sistema
    \item Procure ajuda profissional quando necessário
    \item Respeite todos os direitos trabalhistas
    \item Mantenha documentação organizada
\end{itemize}

\section{Documentos necessários}

\subsection{Do empregado}
\begin{itemize}
    \item CPF
    \item RG ou CNH
    \item Carteira de trabalho
    \item PIS/PASEP
    \item Comprovante de residência
    \item Certidão de nascimento/casamento
    \item Cartão de vacinação (se exigido)
    \item Exames médicos admissionais
\end{itemize}

\subsection{Dos dependentes}
\begin{itemize}
    \item CPF (se tiver)
    \item Certidão de nascimento
    \item Comprovante de matrícula escolar (se estudante)
    \item Cartão de vacinação (menores)
\end{itemize}

\section{Contato NAF}

Para auxílio com o e-Social Doméstico, procure o Núcleo de Apoio Fiscal:

\begin{itemize}
    \item \textbf{Endereço:} Estácio Florianópolis
    \item \textbf{Telefone:} (48) 98461-4449
    \item \textbf{E-mail:} naf@estacio.br
    \item \textbf{Atendimento:} Segunda a sexta, das 8h às 18h
\end{itemize}

\vfill
\begin{center}
\footnotesize
Este guia foi elaborado pelo NAF - Núcleo de Apoio Fiscal da Estácio Florianópolis\\
Última atualização: \today
\end{center}

\end{document}