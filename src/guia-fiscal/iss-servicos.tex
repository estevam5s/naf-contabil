\documentclass[12pt,a4paper]{article}
\usepackage[utf8]{inputenc}
\usepackage[portuguese]{babel}
\usepackage{geometry}
\usepackage{fancyhdr}
\usepackage{graphicx}
\usepackage{xcolor}
\usepackage{hyperref}
\usepackage{enumitem}
\usepackage{titlesec}

\geometry{margin=2.5cm}

\pagestyle{fancy}
\fancyhf{}
\fancyhead[L]{NAF - Núcleo de Apoio Fiscal}
\fancyhead[R]{Estácio Florianópolis}
\fancyfoot[C]{\thepage}

\titleformat{\section}{\large\bfseries\color{blue!60!black}}{\thesection}{1em}{}
\titleformat{\subsection}{\normalsize\bfseries\color{blue!40!black}}{\thesubsection}{1em}{}

\begin{document}

\begin{center}
{\LARGE \textbf{GUIA COMPLETO}}\\[0.5cm]
{\Large \textbf{ISS - IMPOSTO SOBRE SERVIÇOS}}\\[0.5cm]
{\large Núcleo de Apoio Fiscal - NAF}\\
{\large Estácio Florianópolis}\\[1cm]
\end{center}

\section{O que é o ISS?}

O Imposto Sobre Serviços de Qualquer Natureza (ISS ou ISSQN) é um tributo municipal que incide sobre a prestação de serviços por empresas ou profissionais autônomos. É regulamentado pela Lei Complementar 116/2003 e legislações municipais específicas.

\section{Competência tributária}

\subsection{Município competente}
\begin{itemize}
    \item Local onde o serviço é prestado
    \item Local do estabelecimento prestador (regra geral)
    \item Local do domicílio do tomador (serviços específicos)
    \item Município onde se encontra o bem (alguns casos)
\end{itemize}

\subsection{Exceções importantes}
\begin{itemize}
    \item Construção civil: local da obra
    \item Vigilância: local da execução
    \item Limpeza: local da prestação
    \item Serviços de comunicação: local da prestação
\end{itemize}

\section{Serviços sujeitos ao ISS}

A lista de serviços está definida na Lei Complementar 116/2003:

\subsection{Principais categorias}
\begin{itemize}
    \item Serviços de informática e congêneres
    \item Serviços de pesquisas e desenvolvimento
    \item Serviços de assessoria ou consultoria
    \item Serviços de engenharia e arquitetura
    \item Serviços de medicina e biomedicina
    \item Serviços de veterinária
    \item Serviços de psicologia, psicanálise e terapia
    \item Serviços de terapia ocupacional e fisioterapia
    \item Serviços de nutrição
    \item Serviços de obstetrícia
    \item Serviços de odontologia
    \item Serviços de enfermagem e congêneres
    \item Serviços de farmácia e bioquímica
    \item Serviços funerários
    \item Serviços de advocacia
    \item Serviços notariais e de registro
\end{itemize}

\section{Base de cálculo}

\subsection{Regra geral}
A base de cálculo do ISS é o \textbf{preço do serviço}, que compreende:
\begin{itemize}
    \item Valor cobrado pela prestação do serviço
    \item Materiais fornecidos pelo prestador
    \item Subempreitadas
    \item Qualquer quantia recebida pelo prestador
\end{itemize}

\subsection{Deduções permitidas}
Em alguns casos, podem ser deduzidos:
\begin{itemize}
    \item Materiais aplicados (quando a lei municipal permitir)
    \item Subempreitadas (em alguns municípios)
    \item Descontos incondicionais
    \item Valores de terceiros expressamente discriminados
\end{itemize}

\section{Alíquotas do ISS}

\subsection{Limites legais}
\begin{itemize}
    \item \textbf{Mínima:} 2\% (Lei Complementar 116/2003)
    \item \textbf{Máxima:} 5\% (Constituição Federal)
    \item \textbf{Serviços específicos:} Até 2\% (construção civil, alguns casos)
\end{itemize}

\subsection{Variações por município}
\begin{itemize}
    \item Cada município define suas alíquotas
    \item Podem variar conforme o tipo de serviço
    \item Consultar legislação municipal específica
    \item Possibilidade de alíquotas diferenciadas por porte
\end{itemize}

\section{Contribuintes do ISS}

\subsection{Prestadores de serviços}
\begin{itemize}
    \item Empresas prestadoras de serviços
    \item Profissionais liberais autônomos
    \item Sociedades profissionais
    \item Cooperativas de serviços
    \item MEI prestador de serviços
\end{itemize}

\subsection{Responsáveis tributários}
\begin{itemize}
    \item Tomador de serviços (retenção na fonte)
    \item Órgãos públicos
    \item Empresas que contratam serviços
    \item Administradoras de consórcios
\end{itemize}

\section{Retenção na fonte}

\subsection{Quando ocorre}
\begin{itemize}
    \item Serviços prestados por pessoas jurídicas
    \item Para tomadores específicos (órgãos públicos, empresas)
    \item Conforme legislação municipal
    \item Valores acima de limites estabelecidos
\end{itemize}

\subsection{Alíquotas de retenção}
\begin{itemize}
    \item Geralmente entre 2\% a 5\%
    \item Conforme alíquota municipal
    \item Pode ser diferenciada por atividade
    \item Verificar legislação específica
\end{itemize}

\subsection{Obrigações do tomador}
\begin{itemize}
    \item Reter o ISS na fonte
    \item Recolher aos cofres municipais
    \item Emitir comprovante de retenção
    \item Informar em declarações
\end{itemize}

\section{Regimes de tributação}

\subsection{Simples Nacional}
\begin{itemize}
    \item ISS incluso na guia única (DAS)
    \item Alíquotas progressivas por faturamento
    \item Menos burocracia
    \item Para ME e EPP
\end{itemize}

\subsection{Lucro Presumido/Real}
\begin{itemize}
    \item ISS calculado separadamente
    \item Apuração mensal
    \item Recolhimento até dia 15
    \item Maior complexidade contábil
\end{itemize}

\subsection{Profissionais Autônomos}
\begin{itemize}
    \item ISS fixo anual (em alguns municípios)
    \item ISS variável sobre receitas
    \item Livro registro de receitas
    \item Inscrição municipal obrigatória
\end{itemize}

\section{Inscrição municipal}

\subsection{Quando obrigatória}
\begin{itemize}
    \item Empresas prestadoras de serviços
    \item Profissionais autônomos (conforme município)
    \item Estabelecimentos no município
    \item Prestação regular de serviços
\end{itemize}

\subsection{Documentos necessários}
\begin{itemize}
    \item CNPJ (empresas) ou CPF (autônomos)
    \item Contrato social registrado
    \item Comprovante de endereço
    \item Alvará de funcionamento
    \item Documentos pessoais dos sócios
\end{itemize}

\section{Obrigações acessórias}

\subsection{Livros fiscais}
\begin{itemize}
    \item Livro de Registro de Prestação de Serviços
    \item Livro de Registro de Notas Fiscais
    \item Livro de Apuração do ISS
    \item Escrituração contábil (empresas)
\end{itemize}

\subsection{Declarações}
\begin{itemize}
    \item Declaração mensal de serviços prestados
    \item Declaração anual de movimento econômico
    \item DEFIS (Declaração de Informações Socioeconômicas)
    \item Outras conforme município
\end{itemize}

\subsection{Notas fiscais}
\begin{itemize}
    \item Nota Fiscal de Serviços Eletrônica (NFSe)
    \item Emissão obrigatória para pessoas jurídicas
    \item Facultativa para consumidor final (conforme município)
    \item Prazo de emissão: até 5 dias da prestação
\end{itemize}

\section{Nota Fiscal de Serviços Eletrônica (NFSe)}

\subsection{Características}
\begin{itemize}
    \item Documento eletrônico
    \item Substitui nota fiscal em papel
    \item Validade jurídica garantida
    \item Facilita controle fiscal
\end{itemize}

\subsection{Obrigatoriedade}
\begin{itemize}
    \item Pessoas jurídicas prestadoras
    \item Serviços para outras empresas
    \item Serviços para órgãos públicos
    \item Conforme exigência municipal
\end{itemize}

\subsection{Como emitir}
\begin{itemize}
    \item Sistema da prefeitura
    \item Software certificado
    \item Aplicativo móvel (alguns municípios)
    \item Integração com sistema próprio
\end{itemize}

\section{Apuração e recolhimento}

\subsection{Periodicidade}
\begin{itemize}
    \item Mensal (maioria dos municípios)
    \item Quinzenal (alguns casos)
    \item Trimestral (microempresas, alguns municípios)
    \item Anual (ISS fixo)
\end{itemize}

\subsection{Prazo de recolhimento}
\begin{itemize}
    \item Até o dia 15 do mês seguinte (regra geral)
    \item Varia conforme município
    \item Atenção aos feriados e fins de semana
    \item Consultar calendário fiscal municipal
\end{itemize}

\subsection{Forma de pagamento}
\begin{itemize}
    \item DAM (Documento de Arrecadação Municipal)
    \item Carnê-leão (profissionais autônomos)
    \item Débito automático
    \item Internet banking
\end{itemize}

\section{Isenções e imunidades}

\subsection{Imunidades constitucionais}
\begin{itemize}
    \item Exportação de serviços
    \item Livros, jornais e periódicos
    \item Entidades religiosas
    \item Partidos políticos
    \item Sindicatos de trabalhadores
    \item Instituições de assistência social
\end{itemize}

\subsection{Isenções municipais}
\begin{itemize}
    \item Microempresários (conforme lei municipal)
    \item Atividades culturais
    \item Serviços sociais
    \item Cooperativas específicas
    \item Pequenos prestadores (limites municipais)
\end{itemize}

\section{Fiscalização e penalidades}

\subsection{Infrações comuns}
\begin{itemize}
    \item Falta de inscrição municipal
    \item Não emissão de nota fiscal
    \item Atraso no recolhimento
    \item Escrituração irregular
    \item Informações incorretas
\end{itemize}

\subsection{Penalidades aplicáveis}
\begin{itemize}
    \item Multa de mora: 0,33\% ao dia (limite 20\%)
    \item Juros de mora: Taxa Selic
    \item Multa por infração: 50\% a 300\% do tributo
    \item Multa fixa: Por descumprimento formal
    \item Inscrição em dívida ativa
\end{itemize}

\section{Planejamento tributário}

\subsection{Estratégias legais}
\begin{itemize}
    \item Escolha do regime tributário adequado
    \item Análise de municípios para prestação
    \item Estruturação de contratos
    \item Aproveitamento de isenções
    \item Compensação de créditos
\end{itemize}

\subsection{Cuidados necessários}
\begin{itemize}
    \item Verificar legislação específica
    \item Consultar jurisprudência
    \item Documentar decisões
    \item Acompanhar mudanças legislativas
    \item Procurar orientação profissional
\end{itemize}

\section{Dicas importantes}

\begin{itemize}
    \item Mantenha inscrição municipal atualizada
    \item Emita notas fiscais corretamente
    \item Recolha o ISS sempre em dia
    \item Escriture livros fiscais adequadamente
    \item Guarde documentos por 5 anos
    \item Acompanhe mudanças na legislação
    \item Considere impacto do ISS nos preços
    \item Procure orientação contábil especializada
\end{itemize}

\section{Diferenças entre municípios}

\begin{itemize}
    \item Alíquotas podem variar
    \item Obrigações acessórias diferentes
    \item Prazos de recolhimento distintos
    \item Sistemas de NFSe próprios
    \item Benefícios fiscais específicos
    \item Consulte sempre a legislação local
\end{itemize}

\section{Contato NAF}

Para auxílio com ISS - Imposto Sobre Serviços, procure o Núcleo de Apoio Fiscal:

\begin{itemize}
    \item \textbf{Endereço:} Estácio Florianópolis
    \item \textbf{Telefone:} (48) 98461-4449
    \item \textbf{E-mail:} naf@estacio.br
    \item \textbf{Atendimento:} Segunda a sexta, das 8h às 18h
\end{itemize}

\vfill
\begin{center}
\footnotesize
Este guia foi elaborado pelo NAF - Núcleo de Apoio Fiscal da Estácio Florianópolis\\
Última atualização: \today
\end{center}

\end{document>}