\documentclass[12pt,a4paper]{article}
\usepackage[utf8]{inputenc}
\usepackage[portuguese]{babel}
\usepackage{geometry}
\usepackage{fancyhdr}
\usepackage{graphicx}
\usepackage{xcolor}
\usepackage{hyperref}
\usepackage{enumitem}
\usepackage{titlesec}
\usepackage{longtable}

\geometry{margin=2.5cm}

\pagestyle{fancy}
\fancyhf{}
\fancyhead[L]{NAF - Núcleo de Apoio Fiscal}
\fancyhead[R]{Estácio Florianópolis}
\fancyfoot[C]{\thepage}

\titleformat{\section}{\large\bfseries\color{blue!60!black}}{\thesection}{1em}{}
\titleformat{\subsection}{\normalsize\bfseries\color{blue!40!black}}{\thesubsection}{1em}{}

\begin{document}

\begin{center}
{\LARGE \textbf{GUIA COMPLETO}}\\[0.5cm]
{\Large \textbf{DECLARAÇÃO DE IMPOSTO DE RENDA}}\\[0.3cm]
{\Large \textbf{PESSOA FÍSICA}}\\[0.5cm]
{\large Núcleo de Apoio Fiscal - NAF}\\
{\large Estácio Florianópolis}\\[1cm]
\end{center}

\section{Quem deve declarar o Imposto de Renda?}

\subsection{É obrigatório declarar se você:}
\begin{itemize}
    \item Recebeu rendimentos tributáveis acima de R\$ 30.639,90 em 2023
    \item Recebeu rendimentos isentos acima de R\$ 200.000,00
    \item Obteve ganho de capital na alienação de bens ou direitos
    \item Realizou operações em bolsa de valores
    \item Teve receita bruta anual superior a R\$ 153.199,50 em atividade rural
    \item Teve posse ou propriedade de bens acima de R\$ 800.000,00 em 31/12/2023
    \item Passou à condição de residente no Brasil
\end{itemize}

\section{Prazo para declarar}

\begin{itemize}
    \item \textbf{Início:} 15 de março
    \item \textbf{Prazo final:} 31 de maio
    \item \textbf{Retificação:} Até 5 anos após o prazo original
\end{itemize}

\section{Documentos necessários}

\subsection{Documentos pessoais}
\begin{itemize}
    \item CPF do declarante e dependentes
    \item RG ou documento oficial com foto
    \item Título de eleitor
    \item Declaração do ano anterior (se houver)
\end{itemize}

\subsection{Rendimentos}
\begin{itemize}
    \item Informe de Rendimentos do empregador
    \item Comprovantes de rendimentos de autônomos
    \item Extratos de aplicações financeiras
    \item Informe de rendimentos de aluguéis
    \item Comprovantes de pensão alimentícia recebida
\end{itemize}

\subsection{Despesas dedutíveis}
\begin{itemize}
    \item Recibos médicos e odontológicos
    \item Comprovantes de gastos com educação
    \item Recibos de pensão alimentícia paga
    \item Comprovantes de contribuição previdenciária
    \item Carnê-leão (se houver)
\end{itemize}

\subsection{Bens e direitos}
\begin{itemize}
    \item Escrituras de imóveis
    \item Documentos de veículos
    \item Extratos bancários (saldo em 31/12)
    \item Comprovantes de aplicações financeiras
    \item Documentos de outros bens
\end{itemize}

\section{Modelos de declaração}

\subsection{Declaração Simplificada}
\begin{itemize}
    \item Desconto padrão de 20\% dos rendimentos tributáveis
    \item Limite máximo de R\$ 16.754,34
    \item Não permite deduções específicas
    \item Mais simples de preencher
\end{itemize}

\subsection{Declaração Completa}
\begin{itemize}
    \item Permite todas as deduções legais
    \item Necessário comprovar todos os gastos
    \item Indicada quando as deduções superam 20\% da renda
    \item Mais trabalhosa, mas pode resultar em maior restituição
\end{itemize}

\section{Principais deduções}

\begin{longtable}{|p{4cm}|p{4cm}|p{6cm}|}
\hline
\textbf{Tipo} & \textbf{Limite Anual} & \textbf{Observações} \\
\hline
Dependentes & R\$ 2.275,08 por dependente & Sem limite de quantidade \\
\hline
Gastos médicos & Sem limite & Só próprios, dependentes e alimentandos \\
\hline
Educação & R\$ 3.561,50 por pessoa & Ensino infantil ao superior \\
\hline
Pensão alimentícia & Sem limite & Apenas determinada judicialmente \\
\hline
Previdência oficial & Sem limite & INSS e regimes próprios \\
\hline
Previdência privada & 12\% da renda tributável & PGBL e seguros de vida \\
\hline
\end{longtable}

\section{Como preencher a declaração}

\subsection{Programa gerador}
\begin{enumerate}
    \item Baixe o programa no site da Receita Federal
    \item Instale em seu computador
    \item Inicie uma nova declaração
    \item Preencha os dados pessoais
    \item Importe dados se disponíveis
    \item Complete as fichas necessárias
    \item Verifique os cálculos
    \item Transmita pela internet
\end{enumerate}

\subsection{Declaração online}
\begin{enumerate}
    \item Acesse e-CAC da Receita Federal
    \item Faça login com certificado digital ou conta gov.br
    \item Selecione "Meu Imposto de Renda"
    \item Escolha "Declarar online"
    \item Preencha os dados
    \item Envie a declaração
\end{enumerate}

\section{Principais fichas da declaração}

\subsection{Ficha "Rendimentos Tributáveis"}
\begin{itemize}
    \item Salários e pró-labore
    \item Aposentadorias e pensões
    \item Rendimentos de autônomos
    \item Aluguéis recebidos
    \item Outros rendimentos tributáveis
\end{itemize}

\subsection{Ficha "Rendimentos Isentos"}
\begin{itemize}
    \item Rendimentos de caderneta de poupança
    \item Indenizações trabalhistas
    \item Bolsa de estudo
    \item Seguro-desemprego
    \item Outros rendimentos não tributáveis
\end{itemize}

\subsection{Ficha "Dependentes"}
\begin{itemize}
    \item Filhos menores de 21 anos
    \item Filhos estudantes menores de 24 anos
    \item Cônjuge sem rendimentos próprios
    \item Pais e avós sem rendimentos próprios
\end{itemize}

\subsection{Ficha "Bens e Direitos"}
\begin{itemize}
    \item Imóveis
    \item Veículos
    \item Aplicações financeiras
    \item Participações societárias
    \item Outros bens
\end{itemize}

\section{Restituição e imposto a pagar}

\subsection{Restituição}
\begin{itemize}
    \item Ocorre quando foi retido mais imposto que o devido
    \item Pagamento em 5 lotes de maio a setembro
    \item Correção pela taxa Selic
    \item Consulta no site da Receita Federal
\end{itemize}

\subsection{Imposto a pagar}
\begin{itemize}
    \item Pagamento até o prazo de entrega
    \item Parcelamento em até 60 vezes (mínimo R\$ 50,00 por parcela)
    \item Juros de 1\% ao mês
    \item DARF gerado no próprio programa
\end{itemize}

\section{Malha fina}

\subsection{Principais motivos}
\begin{itemize}
    \item Omissão de rendimentos
    \item Deduções sem comprovação
    \item Divergências nos dados informados
    \item Inconsistências nos valores
    \item Gastos incompatíveis com a renda
\end{itemize}

\subsection{Como evitar}
\begin{itemize}
    \item Confira todos os dados antes de enviar
    \item Mantenha documentos organizados
    \item Declare todos os rendimentos
    \item Comprove todas as deduções
    \item Use informações exatas dos informes
\end{itemize}

\section{Retificação}

\subsection{Quando fazer}
\begin{itemize}
    \item Erro no preenchimento
    \item Esquecimento de rendimentos ou deduções
    \item Mudança de situação (casamento, dependentes)
    \item Recebimento de novos documentos
\end{itemize}

\subsection{Como fazer}
\begin{itemize}
    \item Use o mesmo programa da declaração original
    \item Marque "Declaração retificadora"
    \item Faça as correções necessárias
    \item Transmita normalmente
    \item Guarde o recibo
\end{itemize}

\section{Declaração em atraso}

\subsection{Multas}
\begin{itemize}
    \item Mínimo: R\$ 165,74
    \item Máximo: 20\% do imposto devido
    \item Cálculo: 1\% ao mês sobre o imposto devido
    \item Juros Selic sobre a multa
\end{itemize}

\subsection{Como regularizar}
\begin{itemize}
    \item Faça a declaração o quanto antes
    \item Pague a multa junto com o imposto (se houver)
    \item Use DARF para pagamento
    \item Mantenha comprovantes
\end{itemize}

\section{Dicas importantes}

\begin{itemize}
    \item Organize os documentos durante todo o ano
    \item Faça backup da declaração
    \item Confira os cálculos antes de transmitir
    \item Guarde cópia da declaração enviada
    \item Acompanhe o processamento no e-CAC
    \item Mantenha documentos por 5 anos
    \item Declare sempre a verdade
\end{itemize}

\section{Contato NAF}

Para auxílio no preenchimento da declaração de IR, procure o Núcleo de Apoio Fiscal:

\begin{itemize}
    \item \textbf{Endereço:} Estácio Florianópolis
    \item \textbf{Telefone:} (48) 98461-4449
    \item \textbf{E-mail:} naf@estacio.br
    \item \textbf{Atendimento:} Segunda a sexta, das 8h às 18h
    \item \textbf{Período especial:} Março a maio - horário estendido
\end{itemize}

\vfill
\begin{center}
\footnotesize
Este guia foi elaborado pelo NAF - Núcleo de Apoio Fiscal da Estácio Florianópolis\\
Última atualização: \today
\end{center}

\end{document}