\documentclass[12pt,a4paper]{article}
\usepackage[utf8]{inputenc}
\usepackage[portuguese]{babel}
\usepackage{geometry}
\usepackage{fancyhdr}
\usepackage{graphicx}
\usepackage{xcolor}
\usepackage{hyperref}
\usepackage{enumitem}
\usepackage{titlesec}
\usepackage{longtable}

\geometry{margin=2.5cm}

\pagestyle{fancy}
\fancyhf{}
\fancyhead[L]{NAF - Núcleo de Apoio Fiscal}
\fancyhead[R]{Estácio Florianópolis}
\fancyfoot[C]{\thepage}

\titleformat{\section}{\large\bfseries\color{blue!60!black}}{\thesection}{1em}{}
\titleformat{\subsection}{\normalsize\bfseries\color{blue!40!black}}{\thesubsection}{1em}{}

\begin{document}

\begin{center}
{\LARGE \textbf{GUIA COMPLETO}}\\[0.5cm]
{\Large \textbf{ICMS}}\\[0.3cm]
{\Large \textbf{IMPOSTO SOBRE CIRCULAÇÃO DE MERCADORIAS}}\\[0.5cm]
{\large Núcleo de Apoio Fiscal - NAF}\\
{\large Estácio Florianópolis}\\[1cm]
\end{center}

\section{O que é o ICMS?}

O Imposto sobre Operações relativas à Circulação de Mercadorias e sobre Prestações de Serviços de Transporte Interestadual e Intermunicipal e de Comunicação (ICMS) é um tributo estadual que incide sobre a movimentação de mercadorias e alguns serviços específicos.

\section{Competência tributária}

\subsection{Estados e Distrito Federal}
\begin{itemize}
    \item Cada estado tem sua própria legislação
    \item Alíquotas definidas pelos estados
    \item Regulamentação por decretos estaduais
    \item Fiscalização pelos órgãos fazendários estaduais
\end{itemize}

\subsection{Convênios CONFAZ}
\begin{itemize}
    \item Conselho Nacional de Política Fazendária
    \item Uniformização de procedimentos
    \item Concessão de benefícios fiscais
    \item Evita guerra fiscal entre estados
\end{itemize}

\section{Incidência do ICMS}

\subsection{Operações com mercadorias}
\begin{itemize}
    \item Vendas de mercadorias
    \item Transferências entre estabelecimentos
    \item Importação de mercadorias
    \item Fornecimento de alimentação e bebidas
    \item Operações decorrentes de arrendamento mercantil
\end{itemize}

\subsection{Prestações de serviços}
\begin{itemize}
    \item Transporte interestadual e intermunicipal
    \item Comunicações (telefone, internet, TV)
    \item Serviços iniciados no exterior
    \item Energia elétrica
\end{itemize}

\subsection{Outras hipóteses}
\begin{itemize}
    \item Entrada de mercadorias importadas
    \item Utilização de serviços no exterior
    \item Entrada de petróleo e energia elétrica
    \item Operações com ouro (ativo financeiro)
\end{itemize}

\section{Não incidência}

O ICMS não incide sobre:
\begin{itemize}
    \item Operações com livros, jornais e periódicos
    \item Operações de exportação
    \item Operações interestaduais de petróleo e lubrificantes
    \item Operações com ouro (instrumento cambial)
    \item Energia elétrica, petróleo e combustíveis para outros Estados
    \item Arrendamento mercantil (leasing)
\end{itemize}

\section{Contribuintes do ICMS}

\subsection{Contribuintes obrigatórios}
\begin{itemize}
    \item Comerciantes e industriais
    \item Prestadores de serviços de transporte e comunicação
    \item Produtores, extratores e geradores (energia)
    \item Importadores de mercadorias
    \item Cooperativas
\end{itemize}

\subsection{Inscrição estadual}
\begin{itemize}
    \item Obrigatória para contribuintes do ICMS
    \item Número único por estabelecimento
    \item Cadastro na Secretaria da Fazenda Estadual
    \item Renovação periódica conforme estado
\end{itemize}

\section{Base de cálculo}

\subsection{Regra geral}
\begin{itemize}
    \item Valor da operação ou prestação
    \item Valor dos produtos, mercadorias ou serviços
    \item Incluindo frete, seguro e demais despesas
    \item Excluindo descontos incondicionais
\end{itemize}

\subsection{Casos especiais}
\begin{itemize}
    \item Importação: valor aduaneiro + impostos + taxas
    \item Energia elétrica: valor da tarifa
    \item Combustíveis: valor da operação + CIDE
    \item ST (Substituição Tributária): pauta ou preço praticado
\end{itemize}

\section{Alíquotas do ICMS}

\subsection{Operações internas}
\begin{longtable}{|p{6cm}|p{3cm}|p{5cm}|}
\hline
\textbf{Tipo de Produto/Serviço} & \textbf{Alíquota} & \textbf{Observações} \\
\hline
Produtos da cesta básica & 7\% & Varia por estado \\
\hline
Produtos essenciais & 12\% & Regra geral \\
\hline
Produtos normais & 17\% ou 18\% & Conforme estado \\
\hline
Produtos supérfluos & 25\% & Bebidas, fumo, etc. \\
\hline
Energia elétrica & 12\% a 25\% & Conforme consumo \\
\hline
Combustíveis & 25\% a 34\% & Varia por produto \\
\hline
\end{longtable}

\subsection{Operações interestaduais}
\begin{itemize}
    \item Entre contribuintes: 7\% ou 12\%
    \item Para não contribuinte: Alíquota interna do destino
    \item Operações com energia elétrica: Alíquotas específicas
    \item Comunicação: 7\% ou 12\%
\end{itemize}

\section{Regimes de tributação}

\subsection{Simples Nacional}
\begin{itemize}
    \item Alíquotas unificadas incluindo ICMS
    \item Menos obrigações acessórias
    \item Para micro e pequenas empresas
    \item Limitações por atividade e faturamento
\end{itemize}

\subsection{Lucro Real/Presumido}
\begin{itemize}
    \item ICMS apurado separadamente
    \item Regime de débito e crédito
    \item Escrituração fiscal obrigatória
    \item Apuração mensal ou anual
\end{itemize}

\section{Apuração do ICMS}

\subsection{Regime não cumulativo}
\begin{itemize}
    \item ICMS a recolher = Débitos - Créditos
    \item Créditos: ICMS nas aquisições
    \item Débitos: ICMS nas vendas
    \item Saldo credor: Transferir para período seguinte
\end{itemize}

\subsection{Período de apuração}
\begin{itemize}
    \item Mensal (regra geral)
    \item Decendial (alguns produtos específicos)
    \item Semanal (combustíveis em alguns estados)
    \item Por operação (casos específicos)
\end{itemize}

\section{Documentos fiscais}

\subsection{Nota Fiscal Eletrônica (NFe)}
\begin{itemize}
    \item Documento obrigatório para circulação
    \item Emissão antes da saída da mercadoria
    \item Autorização da SEFAZ
    \item DANFE para acompanhar a mercadoria
\end{itemize}

\subsection{Cupom Fiscal Eletrônico (CFe)}
\begin{itemize}
    \item Para vendas ao consumidor final
    \item Emissão através de equipamento SAT
    \item Substitui cupom fiscal ECF
    \item Obrigatório em muitos estados
\end{itemize}

\subsection{Conhecimento de Transporte}
\begin{itemize}
    \item CTe (Conhecimento de Transporte Eletrônico)
    \item Para prestação de serviços de transporte
    \item Documento fiscal próprio
    \item Vinculação com NFe de mercadorias
\end{itemize}

\section{Substituição tributária}

\subsection{Conceito}
\begin{itemize}
    \item Responsabilidade atribuída a terceiro
    \item Recolhimento antecipado do ICMS
    \item Cálculo por dentro (MVA - Margem de Valor Agregado)
    \item Aplicável a produtos específicos
\end{itemize}

\subsection{Produtos sujeitos}
\begin{itemize}
    \item Combustíveis e lubrificantes
    \item Bebidas e refrigerantes
    \item Produtos de perfumaria
    \item Medicamentos
    \item Cigarros
    \item Materiais de construção
    \item Produtos eletrônicos
\end{itemize}

\subsection{Responsáveis}
\begin{itemize}
    \item Indústrias (regra geral)
    \item Distribuidores
    \item Importadores
    \item Refinarias de petróleo
\end{itemize}

\section{Obrigações acessórias}

\subsection{Livros fiscais}
\begin{itemize}
    \item Registro de Entradas
    \item Registro de Saídas
    \item Registro de Inventário
    \item Registro de Apuração do ICMS
    \item Livro de Movimentação de Combustíveis
\end{itemize}

\subsection{Declarações}
\begin{itemize}
    \item GIA (Guia de Informação e Apuração)
    \item SPED Fiscal (EFD)
    \item DMA (Declaração de Mercadorias em Armazéns)
    \item Outras conforme estado
\end{itemize}

\subsection{SPED Fiscal}
\begin{itemize}
    \item Escrituração Fiscal Digital
    \item Arquivo digital mensal
    \item Substitui livros fiscais em papel
    \item Entrega até o dia 15 do mês seguinte
\end{itemize}

\section{Créditos de ICMS}

\subsection{Direito ao crédito}
\begin{itemize}
    \item Mercadorias para revenda
    \item Matérias-primas para industrialização
    \item Material de embalagem
    \item Energia elétrica e combustíveis (uso industrial)
    \item Bens do ativo imobilizado
\end{itemize}

\subsection{Vedações ao crédito}
\begin{itemize}
    \item Saídas isentas ou não tributadas
    \item Uso ou consumo próprio
    \item Energia elétrica, combustíveis (uso geral)
    \item Mercadorias não escrituradas
    \item Operações com contribuintes suspensos
\end{itemize}

\section{Benefícios fiscais}

\subsection{Tipos de benefícios}
\begin{itemize}
    \item Isenção: Não incidência do imposto
    \item Redução de base de cálculo
    \item Diferimento: Postergação do pagamento
    \item Crédito presumido
    \item Suspensão: Para operações específicas
\end{itemize}

\subsection{Principais benefícios}
\begin{itemize}
    \item Produtos da cesta básica
    \item Medicamentos e equipamentos médicos
    \item Livros e material escolar
    \item Produtos destinados à exportação
    \item Operações do agronegócio
\end{itemize}

\section{Fiscalização}

\subsection{Órgãos fiscalizadores}
\begin{itemize}
    \item Secretarias de Fazenda Estaduais
    \item Receita Federal (comércio exterior)
    \item Polícia Fazendária
    \item Delegacias Tributárias
\end{itemize}

\subsection{Procedimentos fiscais}
\begin{itemize}
    \item Intimação fiscal
    \item Apreensão de mercadorias
    \item Lacre de estabelecimentos
    \item Auto de infração
    \item Processo administrativo
\end{itemize}

\section{Penalidades}

\subsection{Infrações comuns}
\begin{itemize}
    \item Falta de inscrição estadual
    \item Não emissão de documento fiscal
    \item Escrituração incorreta
    \item Atraso no recolhimento
    \item Transporte sem documentação
\end{itemize}

\subsection{Multas aplicáveis}
\begin{itemize}
    \item Multa de mora: 0,33\% ao dia
    \item Juros: Taxa Selic
    \item Multa formal: Por descumprimento
    \item Multa material: 37,5\% a 300\% do tributo
    \item Perdimento de mercadorias
\end{itemize}

\section{Diferencial de alíquotas}

\subsection{Quando ocorre}
\begin{itemize}
    \item Compras interestaduais para uso/consumo
    \item Aquisição de não contribuinte
    \item Diferença entre alíquotas interna e interestadual
    \item Recolhimento no estado de destino
\end{itemize}

\subsection{Cálculo}
Diferencial = (Alíquota interna - Alíquota interestadual) × Base de cálculo

\section{Dicas importantes}

\begin{itemize}
    \item Mantenha inscrição estadual atualizada
    \item Emita documentos fiscais corretamente
    \item Escriture livros fiscais adequadamente
    \item Recolha ICMS sempre em dia
    \item Aproveite créditos legitimamente
    \item Acompanhe mudanças na legislação
    \item Mantenha documentação organizada
    \item Procure orientação contábil especializada
    \item Considere benefícios fiscais disponíveis
    \item Planeje operações interestaduais
\end{itemize}

\section{Planejamento tributário}

\begin{itemize}
    \item Análise de enquadramento fiscal
    \item Aproveitamento de benefícios
    \item Estruturação de operações
    \item Otimização de créditos
    \item Escolha de fornecedores
    \item Localização estratégica
    \item Gestão de estoques
    \item Timing das operações
\end{itemize}

\section{Contato NAF}

Para auxílio com ICMS - Imposto sobre Circulação de Mercadorias, procure o Núcleo de Apoio Fiscal:

\begin{itemize}
    \item \textbf{Endereço:} Estácio Florianópolis
    \item \textbf{Telefone:} (48) 98461-4449
    \item \textbf{E-mail:} naf@estacio.br
    \item \textbf{Atendimento:} Segunda a sexta, das 8h às 18h
\end{itemize}

\vfill
\begin{center}
\footnotesize
Este guia foi elaborado pelo NAF - Núcleo de Apoio Fiscal da Estácio Florianópolis\\
Última atualização: \today
\end{center}

\end{document}