\documentclass[12pt,a4paper]{article}
\usepackage[utf8]{inputenc}
\usepackage[portuguese]{babel}
\usepackage{geometry}
\usepackage{fancyhdr}
\usepackage{graphicx}
\usepackage{xcolor}
\usepackage{hyperref}
\usepackage{enumitem}
\usepackage{titlesec}

\geometry{margin=2.5cm}

\pagestyle{fancy}
\fancyhf{}
\fancyhead[L]{NAF - Núcleo de Apoio Fiscal}
\fancyhead[R]{Estácio Florianópolis}
\fancyfoot[C]{\thepage}

\titleformat{\section}{\large\bfseries\color{blue!60!black}}{\thesection}{1em}{}
\titleformat{\subsection}{\normalsize\bfseries\color{blue!40!black}}{\thesubsection}{1em}{}

\begin{document}

\begin{center}
{\LARGE \textbf{GUIA COMPLETO}}\\[0.5cm]
{\Large \textbf{ALVARÁ DE FUNCIONAMENTO}}\\[0.3cm]
{\Large \textbf{MUNICIPAL}}\\[0.5cm]
{\large Núcleo de Apoio Fiscal - NAF}\\
{\large Estácio Florianópolis}\\[1cm]
\end{center}

\section{O que é o Alvará de Funcionamento?}

O Alvará de Funcionamento é uma licença obrigatória emitida pela prefeitura municipal que autoriza o estabelecimento comercial, industrial ou de prestação de serviços a funcionar em determinado endereço, atestando que o local está em conformidade com a legislação municipal.

\section{Por que é obrigatório?}

\subsection{Fundamentos legais}
\begin{itemize}
    \item Código Tributário Municipal
    \item Lei de Uso e Ocupação do Solo
    \item Código de Obras municipal
    \item Lei de Zoneamento urbano
    \item Código Sanitário municipal
\end{itemize}

\subsection{Finalidades}
\begin{itemize}
    \item Controlar atividades econômicas no município
    \item Garantir conformidade com zoneamento urbano
    \item Verificar condições de segurança
    \item Assegurar salubridade do estabelecimento
    \item Facilitar fiscalização municipal
    \item Proteger direitos da vizinhança
\end{itemize}

\section{Quem precisa do Alvará?}

\subsection{Estabelecimentos obrigatórios}
\begin{itemize}
    \item Lojas comerciais
    \item Escritórios de serviços
    \item Consultórios profissionais
    \item Restaurantes e lanchonetes
    \item Oficinas mecânicas
    \item Salões de beleza
    \item Academias e clínicas
    \item Indústrias e fábricas
    \item Depósitos e armazéns
    \item Estabelecimentos de ensino
    \item Hotéis e pousadas
    \item Postos de combustível
\end{itemize}

\subsection{Exceções}
\begin{itemize}
    \item Profissionais autônomos em domicílio
    \item Atividades rurais (ITR)
    \item Vendedores ambulantes (licença específica)
    \item Home offices sem movimentação de clientes
\end{itemize}

\section{Tipos de Alvará}

\subsection{Alvará Provisório}
\begin{itemize}
    \item Válido por 180 dias (renovável)
    \item Emitido mais rapidamente
    \item Permite início das atividades
    \item Sujeito a vistoria posterior
    \item Pode ser convertido em definitivo
\end{itemize}

\subsection{Alvará Definitivo}
\begin{itemize}
    \item Válido por tempo indeterminado
    \item Requer vistoria prévia
    \item Processo mais demorado
    \item Maior segurança jurídica
    \item Renovação periódica conforme município
\end{itemize}

\subsection{Alvará Simplificado}
\begin{itemize}
    \item Para atividades de baixo risco
    \item Processo automatizado
    \item Emissão online imediata
    \item Lista específica de CNAEs
    \item Menor burocracia
\end{itemize}

\section{Documentos necessários}

\subsection{Documentos da empresa}
\begin{itemize}
    \item CNPJ atualizado
    \item Contrato social registrado
    \item Cartão CNPJ (Receita Federal)
    \item Inscrição municipal
    \item Certidão negativa de débitos municipais
\end{itemize}

\subsection{Documentos do imóvel}
\begin{itemize}
    \item Escritura ou contrato de locação
    \item IPTU atualizado
    \item Planta aprovada do imóvel
    \item Habite-se ou certificado de conclusão
    \item Auto de vistoria do Corpo de Bombeiros (se necessário)
\end{itemize}

\subsection{Documentos do representante}
\begin{itemize}
    \item CPF e RG do responsável
    \item Comprovante de residência
    \item Procuração (se representado por terceiro)
\end{itemize}

\subsection{Documentos específicos}
\begin{itemize}
    \item Licença sanitária (atividades alimentícias)
    \item Licença ambiental (atividades poluidoras)
    \item Autorização de funcionamento (atividades regulamentadas)
    \item Laudo técnico (atividades específicas)
\end{itemize}

\section{Processo de obtenção}

\subsection{1ª Etapa: Consulta prévia}
\begin{enumerate}
    \item Verificar zoneamento do endereço
    \item Consultar CNAE permitido
    \item Confirmar documentos necessários
    \item Verificar taxas municipais
\end{enumerate}

\subsection{2ª Etapa: Preparação}
\begin{enumerate}
    \item Reunir toda documentação
    \item Preencher formulários
    \item Preparar o estabelecimento
    \item Providenciar licenças complementares
\end{enumerate}

\subsection{3ª Etapa: Protocolo}
\begin{enumerate}
    \item Protocolar pedido na prefeitura
    \item Pagar taxas municipais
    \item Receber número de protocolo
    \item Aguardar análise documental
\end{enumerate}

\subsection{4ª Etapa: Vistoria}
\begin{enumerate}
    \item Aguardar agendamento da vistoria
    \item Preparar estabelecimento
    \item Acompanhar fiscal durante vistoria
    \item Corrigir pendências se houver
\end{enumerate}

\subsection{5ª Etapa: Emissão}
\begin{enumerate}
    \item Aguardar análise final
    \item Retirar alvará aprovado
    \item Afixar no estabelecimento
    \item Iniciar atividades legalmente
\end{enumerate}

\section{Consulta de Viabilidade}

\subsection{Importância}
\begin{itemize}
    \item Evita investimentos desnecessários
    \item Confirma legalidade da atividade
    \item Identifica restrições de zoneamento
    \item Previne problemas futuros
    \item Orienta escolha do local
\end{itemize}

\subsection{Como fazer}
\begin{itemize}
    \item Site da prefeitura municipal
    \item Atendimento presencial
    \item Consulta por telefone
    \item Através de despachante
    \item Consulta ao plano diretor
\end{itemize}

\subsection{Informações fornecidas}
\begin{itemize}
    \item CNAEs permitidos no endereço
    \item Restrições de horário
    \item Limitações de área
    \item Exigências especiais
    \item Licenças adicionais necessárias
\end{itemize}

\section{Itens verificados na vistoria}

\subsection{Estrutura física}
\begin{itemize}
    \item Condições gerais do imóvel
    \item Ventilação e iluminação adequadas
    \item Instalações elétricas seguras
    \item Instalações hidráulicas funcionais
    \item Acessibilidade para deficientes
    \item Saídas de emergência
\end{itemize}

\subsection{Segurança}
\begin{itemize}
    \item Extintores de incêndio
    \item Sinalização de segurança
    \item Equipamentos de proteção
    \item Primeiros socorros
    \item Plano de evacuação
\end{itemize}

\subsection{Sanitárias}
\begin{itemize}
    \item Banheiros adequados
    \item Limpeza do estabelecimento
    \item Destino do lixo
    \item Controle de pragas
    \item Ventilação adequada
\end{itemize}

\subsection{Específicas da atividade}
\begin{itemize}
    \item Equipamentos necessários
    \item Licenças profissionais
    \item Normas técnicas específicas
    \item Controles ambientais
    \item Registros obrigatórios
\end{itemize}

\section{Custos envolvidos}

\subsection{Taxas municipais (variam por cidade)}
\begin{itemize}
    \item Taxa de licença: R\$ 50 a R\$ 500
    \item Taxa de vistoria: R\$ 30 a R\$ 200
    \item Taxa de expediente: R\$ 20 a R\$ 100
    \item Taxa de publicação: R\$ 30 a R\$ 150
\end{itemize}

\subsection{Custos adicionais}
\begin{itemize}
    \item Licenças complementares: R\$ 100 a R\$ 1.000
    \item Adequações no local: R\$ 500 a R\$ 10.000
    \item Despachante: R\$ 300 a R\$ 1.500
    \item Consultoria: R\$ 200 a R\$ 2.000
\end{itemize}

\section{Prazos médios}

\begin{itemize}
    \item \textbf{Análise documental:} 15 a 30 dias
    \item \textbf{Agendamento da vistoria:} 10 a 30 dias
    \item \textbf{Correção de pendências:} 15 a 45 dias
    \item \textbf{Emissão final:} 5 a 15 dias
    \item \textbf{Total médio:} 45 a 120 dias
\end{itemize}

\section{Renovação do Alvará}

\subsection{Periodicidade}
\begin{itemize}
    \item Varia conforme município (1 a 5 anos)
    \item Alguns são permanentes
    \item Verificar na legislação local
    \item Observar data de vencimento
\end{itemize}

\subsection{Processo}
\begin{itemize}
    \item Solicitação antes do vencimento
    \item Atualização de dados
    \item Pagamento de taxas
    \item Nova vistoria (se necessário)
    \item Emissão do novo alvará
\end{itemize}

\section{Alterações no Alvará}

\subsection{Quando necessárias}
\begin{itemize}
    \item Mudança de endereço
    \item Alteração de atividade (CNAE)
    \item Ampliação do estabelecimento
    \item Mudança de razão social
    \item Inclusão de novas atividades
\end{itemize}

\subsection{Procedimentos}
\begin{itemize}
    \item Protocolar pedido de alteração
    \item Apresentar documentação atualizada
    \item Pagar taxas devidas
    \item Aguardar nova análise/vistoria
    \item Retirar alvará alterado
\end{itemize}

\section{Fiscalização}

\subsection{Órgãos fiscalizadores}
\begin{itemize}
    \item Fiscalização municipal
    \item Vigilância sanitária
    \item Corpo de bombeiros
    \item Órgãos ambientais
    \item Ministério do Trabalho
\end{itemize}

\subsection{Consequências da falta de alvará}
\begin{itemize}
    \item Multa: R\$ 500 a R\$ 10.000 (varia por município)
    \item Interdição do estabelecimento
    \item Lacração de equipamentos
    \item Apreensão de mercadorias
    \item Processo administrativo
    \item Impossibilidade de renovar licenças
\end{itemize}

\section{Alvará digital}

\subsection{Vantagens}
\begin{itemize}
    \item Processo mais rápido
    \item Menor burocracia
    \item Emissão online
    \item Acompanhamento em tempo real
    \item Menos custos
\end{itemize}

\subsection{Disponibilidade}
\begin{itemize}
    \item Disponível em muitos municípios
    \item Consulte site da prefeitura
    \item Requer certificado digital
    \item Para atividades de baixo risco
\end{itemize}

\section{Dicas importantes}

\begin{itemize}
    \item Consulte sempre antes de alugar ou comprar imóvel
    \item Verifique legislação municipal específica
    \item Mantenha documentação sempre atualizada
    \item Observe prazos de renovação
    \item Comunique alterações à prefeitura
    \item Afixe o alvará em local visível
    \item Mantenha cópia de segurança
    \item Procure orientação profissional quando necessário
\end{itemize}

\section{Erros comuns}

\begin{itemize}
    \item Não consultar viabilidade antes de alugar
    \item Iniciar atividade sem alvará
    \item Não renovar no prazo
    \item Alterar atividade sem comunicar
    \item Descumprir exigências da vistoria
    \item Não manter alvará atualizado
    \item Ignorar notificações municipais
\end{itemize}

\section{Contato NAF}

Para auxílio com Alvará de Funcionamento, procure o Núcleo de Apoio Fiscal:

\begin{itemize}
    \item \textbf{Endereço:} Estácio Florianópolis
    \item \textbf{Telefone:} (48) 98461-4449
    \item \textbf{E-mail:} naf@estacio.br
    \item \textbf{Atendimento:} Segunda a sexta, das 8h às 18h
\end{itemize}

\vfill
\begin{center}
\footnotesize
Este guia foi elaborado pelo NAF - Núcleo de Apoio Fiscal da Estácio Florianópolis\\
Última atualização: \today
\end{center}

\end{document>