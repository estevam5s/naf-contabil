\documentclass[12pt,a4paper]{article}
\usepackage[utf8]{inputenc}
\usepackage[portuguese]{babel}
\usepackage{geometry}
\usepackage{fancyhdr}
\usepackage{graphicx}
\usepackage{xcolor}
\usepackage{hyperref}
\usepackage{enumitem}
\usepackage{titlesec}

\geometry{margin=2.5cm}

\pagestyle{fancy}
\fancyhf{}
\fancyhead[L]{NAF - Núcleo de Apoio Fiscal}
\fancyhead[R]{Estácio Florianópolis}
\fancyfoot[C]{\thepage}

\titleformat{\section}{\large\bfseries\color{blue!60!black}}{\thesection}{1em}{}
\titleformat{\subsection}{\normalsize\bfseries\color{blue!40!black}}{\thesubsection}{1em}{}

\begin{document}

\begin{center}
{\LARGE \textbf{GUIA COMPLETO}}\\[0.5cm]
{\Large \textbf{CADASTRO DE CPF}}\\[0.5cm]
{\large Núcleo de Apoio Fiscal - NAF}\\
{\large Estácio Florianópolis}\\[1cm]
\end{center}

\section{O que é o CPF?}

O Cadastro de Pessoas Físicas (CPF) é um documento de identificação do contribuinte perante a Receita Federal do Brasil. É obrigatório para todas as pessoas físicas residentes no país e é utilizado para controle das atividades relacionadas à tributação.

\section{Quem deve se inscrever no CPF?}

\begin{itemize}
    \item Todas as pessoas físicas residentes no Brasil
    \item Brasileiros residentes no exterior
    \item Estrangeiros que residam no Brasil ou que tenham vínculo com fontes de renda no país
    \item Menores de idade dependentes em declarações de Imposto de Renda
\end{itemize}

\section{Documentos necessários}

\subsection{Para brasileiros}
\begin{itemize}
    \item Certidão de nascimento ou casamento (original ou cópia autenticada)
    \item Documento de identidade oficial com foto (RG, CNH, etc.)
    \item Comprovante de residência atual
\end{itemize}

\subsection{Para estrangeiros}
\begin{itemize}
    \item Documento de identidade do país de origem
    \item Documento que comprove a permanência legal no Brasil
    \item Comprovante de residência no Brasil
\end{itemize}

\section{Como fazer o cadastro}

\subsection{Pela internet}
\begin{enumerate}
    \item Acesse o site da Receita Federal: \url{https://www.receita.fazenda.gov.br}
    \item Clique em "Serviços Online"
    \item Selecione "CPF"
    \item Escolha "Inscrição no CPF"
    \item Preencha todos os dados solicitados
    \item Imprima o recibo de inscrição
\end{enumerate}

\subsection{Presencialmente}
\begin{enumerate}
    \item Procure uma agência dos Correios, Banco do Brasil, Caixa Econômica Federal ou posto de atendimento da Receita Federal
    \item Leve os documentos necessários
    \item Preencha o formulário de inscrição
    \item Pague a taxa (se aplicável)
    \item Aguarde o processamento
\end{enumerate}

\section{Taxas}

\begin{itemize}
    \item \textbf{Primeira inscrição:} Gratuita pela internet
    \item \textbf{Atendimento presencial:} R\$ 7,00 (sujeito a alteração)
    \item \textbf{Segunda via:} R\$ 7,00 (sujeito a alteração)
\end{itemize}

\section{Prazo para recebimento}

\begin{itemize}
    \item \textbf{Pela internet:} Imediato (número provisório), confirmação em até 5 dias úteis
    \item \textbf{Presencial:} Até 5 dias úteis
\end{itemize}

\section{Situação do CPF}

O CPF pode ter as seguintes situações:
\begin{itemize}
    \item \textbf{Regular:} Permite todas as operações
    \item \textbf{Pendente de regularização:} Necessita atualização de dados
    \item \textbf{Suspenso:} Impedido de realizar operações
    \item \textbf{Cancelado:} Inativo por determinação da Receita Federal
\end{itemize}

\section{Regularização do CPF}

Para regularizar um CPF pendente ou suspenso:
\begin{enumerate}
    \item Acesse o Portal e-CAC da Receita Federal
    \item Faça login com sua conta gov.br
    \item Selecione "Meu CPF"
    \item Siga as orientações para regularização
    \item Atualize os dados solicitados
\end{enumerate}

\section{Dicas importantes}

\begin{itemize}
    \item Mantenha sempre seus dados atualizados
    \item Guarde bem o número do seu CPF
    \item Nunca empreste seu CPF para terceiros
    \item Consulte periodicamente a situação do seu CPF
    \item Em caso de perda, solicite segunda via imediatamente
\end{itemize}

\section{Contato NAF}

Para mais informações ou dúvidas, procure o Núcleo de Apoio Fiscal:

\begin{itemize}
    \item \textbf{Endereço:} Estácio Florianópolis
    \item \textbf{Telefone:} (48) 98461-4449
    \item \textbf{E-mail:} naf@estacio.br
    \item \textbf{Atendimento:} Segunda a sexta, das 8h às 18h
\end{itemize}

\vfill
\begin{center}
\footnotesize
Este guia foi elaborado pelo NAF - Núcleo de Apoio Fiscal da Estácio Florianópolis\\
Última atualização: \today
\end{center}

\end{document}