\documentclass[12pt,a4paper]{article}
\usepackage[utf8]{inputenc}
\usepackage[portuguese]{babel}
\usepackage{geometry}
\usepackage{fancyhdr}
\usepackage{graphicx}
\usepackage{xcolor}
\usepackage{hyperref}
\usepackage{enumitem}
\usepackage{titlesec}

\geometry{margin=2.5cm}

\pagestyle{fancy}
\fancyhf{}
\fancyhead[L]{NAF - Núcleo de Apoio Fiscal}
\fancyhead[R]{Estácio Florianópolis}
\fancyfoot[C]{\thepage}

\titleformat{\section}{\large\bfseries\color{blue!60!black}}{\thesection}{1em}{}
\titleformat{\subsection}{\normalsize\bfseries\color{blue!40!black}}{\thesubsection}{1em}{}

\begin{document}

\begin{center}
{\LARGE \textbf{GUIA COMPLETO}}\\[0.5cm]
{\Large \textbf{ABERTURA DE CNPJ}}\\[0.5cm]
{\large Núcleo de Apoio Fiscal - NAF}\\
{\large Estácio Florianópolis}\\[1cm]
\end{center}

\section{O que é CNPJ?}

O Cadastro Nacional da Pessoa Jurídica (CNPJ) é um número único que identifica uma pessoa jurídica junto à Receita Federal. É obrigatório para empresas, associações, fundações e outros tipos de organizações.

\section{Tipos de empresa}

\subsection{MEI - Microempreendedor Individual}
\begin{itemize}
    \item Faturamento até R\$ 81.000/ano
    \item Até 1 funcionário
    \item Processo simplificado
    \item Inscrição gratuita pela internet
\end{itemize}

\subsection{Microempresa (ME)}
\begin{itemize}
    \item Faturamento até R\$ 360.000/ano
    \item Optante pelo Simples Nacional
    \item Tributação simplificada
    \item Menos burocracia
\end{itemize}

\subsection{Empresa de Pequeno Porte (EPP)}
\begin{itemize}
    \item Faturamento de R\$ 360.000 a R\$ 4.800.000/ano
    \item Pode optar pelo Simples Nacional
    \item Benefícios tributários
    \item Maior flexibilidade
\end{itemize}

\subsection{Demais empresas}
\begin{itemize}
    \item Faturamento acima de R\$ 4.800.000/ano
    \item Lucro Real ou Presumido
    \item Tributação mais complexa
    \item Maiores obrigações acessórias
\end{itemize}

\section{Modalidades empresariais}

\subsection{Empresário Individual (EI)}
\begin{itemize}
    \item Uma pessoa física
    \item Responsabilidade ilimitada
    \item Patrimônio pessoal e empresarial se confundem
    \item Não tem sócios
\end{itemize}

\subsection{EIRELI - Empresa Individual de Responsabilidade Limitada}
\begin{itemize}
    \item Uma pessoa física ou jurídica
    \item Responsabilidade limitada ao capital social
    \item Capital mínimo de 100 salários mínimos
    \item Extinção prevista para 2021 (novo marco legal)
\end{itemize}

\subsection{Sociedade Limitada (LTDA)}
\begin{itemize}
    \item Dois ou mais sócios
    \item Responsabilidade limitada ao capital social
    \item Mais comum no Brasil
    \item Flexibilidade na gestão
\end{itemize}

\subsection{Sociedade Anônima (S/A)}
\begin{itemize}
    \item Capital dividido em ações
    \item Pode ser aberta ou fechada
    \item Maior complexidade
    \item Ideal para grandes empreendimentos
\end{itemize}

\section{Documentos necessários}

\subsection{Documentos pessoais dos sócios}
\begin{itemize}
    \item CPF
    \item RG ou outro documento oficial
    \item Comprovante de residência atualizado
    \item Certidão de casamento (se casado)
    \item Declaração de imposto de renda (última)
\end{itemize}

\subsection{Documentos do endereço}
\begin{itemize}
    \item Contrato de locação ou escritura
    \item IPTU do imóvel
    \item Comprovante de residência do proprietário
    \item Autorização do proprietário (se locado)
\end{itemize}

\subsection{Outros documentos}
\begin{itemize}
    \item Consulta de viabilidade de endereço
    \item Consulta de nome empresarial
    \item Contrato social (minuta)
    \item DBE (Declaração de Beneficiário Efetivo)
\end{itemize}

\section{Passo a passo para abertura}

\subsection{1ª Etapa: Junta Comercial}
\begin{enumerate}
    \item Consulta de nome empresarial
    \item Elaboração do contrato social
    \item Protocolo na Junta Comercial
    \item Pagamento das taxas
    \item Aguardar deferimento
    \item Retirar documentos
\end{enumerate}

\subsection{2ª Etapa: Receita Federal}
\begin{enumerate}
    \item Acesso ao Portal da Receita Federal
    \item Preenchimento da ficha cadastral
    \item Upload dos documentos
    \item Transmissão eletrônica
    \item Aguardar processamento
    \item Emissão do CNPJ
\end{enumerate}

\subsection{3ª Etapa: Licenças municipais}
\begin{enumerate}
    \item Consulta de viabilidade de localização
    \item Solicitação de licenças
    \item Vistoria (se necessária)
    \item Pagamento de taxas
    \item Emissão do alvará de funcionamento
\end{enumerate}

\subsection{4ª Etapa: Licenças estaduais}
\begin{enumerate}
    \item Inscrição estadual (se necessária)
    \item Licenças ambientais (se aplicável)
    \item Corpo de Bombeiros (se necessário)
    \item Vigilância Sanitária (se aplicável)
\end{enumerate}

\section{Regime tributário}

\subsection{Simples Nacional}
\begin{itemize}
    \item Para ME e EPP
    \item Tributação unificada
    \item Alíquotas reduzidas
    \item Menos burocracia
    \item Restrições por atividade
\end{itemize}

\subsection{Lucro Presumido}
\begin{itemize}
    \item Base de cálculo presumida
    \item Percentuais fixos por atividade
    \item Menor complexidade contábil
    \item Adequado para margens altas
\end{itemize}

\subsection{Lucro Real}
\begin{itemize}
    \item Base no lucro efetivo
    \item Contabilidade completa
    \item Obrigatório para grandes empresas
    \item Mais vantajoso com prejuízo
\end{itemize}

\section{Custos envolvidos}

\subsection{Taxas obrigatórias}
\begin{itemize}
    \item Junta Comercial: R\$ 100 a R\$ 300
    \item Taxa de expediente: R\$ 40 a R\$ 80
    \item Certidões: R\$ 50 a R\$ 100
    \item Alvará municipal: Varia por município
\end{itemize}

\subsection{Custos opcionais}
\begin{itemize}
    \item Contador: R\$ 200 a R\$ 1.000
    \item Despachante: R\$ 500 a R\$ 2.000
    \item Sede virtual: R\$ 100 a R\$ 500/mês
    \item Consultoria: R\$ 500 a R\$ 5.000
\end{itemize}

\section{Prazos médios}

\begin{itemize}
    \item \textbf{Junta Comercial:} 5 a 15 dias úteis
    \item \textbf{Receita Federal:} 1 a 5 dias úteis
    \item \textbf{Prefeitura:} 15 a 30 dias úteis
    \item \textbf{Estado:} 10 a 30 dias úteis
    \item \textbf{Total:} 30 a 60 dias (média)
\end{itemize}

\section{CNAE - Classificação Nacional de Atividades Econômicas}

\subsection{Importância}
\begin{itemize}
    \item Define a atividade principal da empresa
    \item Determina tributos aplicáveis
    \item Influencia licenças necessárias
    \item Afeta enquadramento no Simples Nacional
\end{itemize}

\subsection{Como escolher}
\begin{itemize}
    \item Identifique sua atividade principal
    \item Consulte a tabela CNAE
    \item Verifique restrições e benefícios
    \item Considere atividades futuras
    \item Procure orientação profissional
\end{itemize}

\section{Obrigações após a abertura}

\subsection{Mensais}
\begin{itemize}
    \item Apuração e pagamento de impostos
    \item Escrituração contábil
    \item Guias de recolhimento
    \item Folha de pagamento (se houver funcionários)
\end{itemize}

\subsection{Anuais}
\begin{itemize}
    \item Declaração de Imposto de Renda Pessoa Jurídica
    \item Declaração do Simples Nacional (se optante)
    \item Relatório anual de informações sociais
    \item Balanço patrimonial
\end{itemize}

\section{Alterações contratuais}

\subsection{Quando necessárias}
\begin{itemize}
    \item Mudança de endereço
    \item Inclusão/exclusão de sócios
    \item Alteração de capital social
    \item Mudança de atividade
    \item Modificação na administração
\end{itemize}

\subsection{Processo}
\begin{itemize}
    \item Elaborar instrumento de alteração
    \item Registrar na Junta Comercial
    \item Atualizar na Receita Federal
    \item Comunicar outros órgãos
    \item Atualizar licenças se necessário
\end{itemize}

\section{Encerramento de empresa}

\subsection{Processo de baixa}
\begin{itemize}
    \item Quitação de débitos
    \item Cancelamento de inscrições
    \item Arquivamento na Junta Comercial
    \item Baixa na Receita Federal
    \item Comunicação aos órgãos
\end{itemize}

\subsection{Documentos necessários}
\begin{itemize}
    \item Distrato social ou ata de dissolução
    \item Certidões negativas
    \item Balanço final
    \item Comprovantes de quitação
\end{itemize}

\section{Dicas importantes}

\begin{itemize}
    \item Pesquise bem antes de escolher o tipo de empresa
    \item Consulte um contador qualificado
    \item Mantenha documentos organizados
    \item Acompanhe prazos e obrigações
    \item Considere custos operacionais
    \item Planeje o negócio adequadamente
    \item Conheça as responsabilidades tributárias
\end{itemize}

\section{Erros comuns}

\begin{itemize}
    \item Escolher regime tributário inadequado
    \item Não consultar viabilidade do endereço
    \item Definir CNAE incorreto
    \item Não considerar custos operacionais
    \item Ignorar licenças municipais/estaduais
    \item Não planejar capital de giro
    \item Descumprir obrigações acessórias
\end{itemize}

\section{Contato NAF}

Para auxílio na abertura de CNPJ, procure o Núcleo de Apoio Fiscal:

\begin{itemize}
    \item \textbf{Endereço:} Estácio Florianópolis
    \item \textbf{Telefone:} (48) 98461-4449
    \item \textbf{E-mail:} naf@estacio.br
    \item \textbf{Atendimento:} Segunda a sexta, das 8h às 18h
\end{itemize}

\vfill
\begin{center}
\footnotesize
Este guia foi elaborado pelo NAF - Núcleo de Apoio Fiscal da Estácio Florianópolis\\
Última atualização: \today
\end{center}

\end{document}