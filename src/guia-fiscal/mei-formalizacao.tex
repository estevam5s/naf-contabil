\documentclass[12pt,a4paper]{article}
\usepackage[utf8]{inputenc}
\usepackage[portuguese]{babel}
\usepackage{geometry}
\usepackage{fancyhdr}
\usepackage{graphicx}
\usepackage{xcolor}
\usepackage{hyperref}
\usepackage{enumitem}
\usepackage{titlesec}
\usepackage{longtable}

\geometry{margin=2.5cm}

\pagestyle{fancy}
\fancyhf{}
\fancyhead[L]{NAF - Núcleo de Apoio Fiscal}
\fancyhead[R]{Estácio Florianópolis}
\fancyfoot[C]{\thepage}

\titleformat{\section}{\large\bfseries\color{blue!60!black}}{\thesection}{1em}{}
\titleformat{\subsection}{\normalsize\bfseries\color{blue!40!black}}{\thesubsection}{1em}{}

\begin{document}

\begin{center}
{\LARGE \textbf{GUIA COMPLETO}}\\[0.5cm]
{\Large \textbf{MEI - FORMALIZAÇÃO E GESTÃO}}\\[0.5cm]
{\large Microempreendedor Individual}\\[0.3cm]
{\large Núcleo de Apoio Fiscal - NAF}\\
{\large Estácio Florianópolis}\\[1cm]
\end{center}

\section{O que é o MEI?}

O Microempreendedor Individual (MEI) é uma modalidade empresarial criada para formalizar trabalhadores autônomos. Permite que pessoas que trabalham por conta própria se tornem pequenos empresários legalizados.

\section{Quem pode ser MEI?}

\subsection{Requisitos obrigatórios}
\begin{itemize}
    \item Ter faturamento anual de até R\$ 81.000,00
    \item Não ter participação em outra empresa como sócio ou titular
    \item Contratar no máximo 1 empregado
    \item Exercer atividade permitida para MEI
\end{itemize}

\subsection{Quem NÃO pode ser MEI}
\begin{itemize}
    \item Profissionais que necessitam de formação específica (médicos, advogados, dentistas, etc.)
    \item Servidores públicos federais em atividade
    \item Pensionistas e beneficiários do INSS (com algumas exceções)
\end{itemize}

\section{Vantagens do MEI}

\begin{itemize}
    \item CNPJ gratuito
    \item Isenção de tributos federais (IR, PIS, COFINS, IPI)
    \item Contribuição mensal reduzida
    \item Emissão de nota fiscal
    \item Acesso a benefícios previdenciários
    \item Facilidade para abrir conta bancária empresarial
    \item Possibilidade de contratar um empregado
\end{itemize}

\section{Como se formalizar}

\subsection{Documentos necessários}
\begin{itemize}
    \item CPF
    \item RG
    \item Comprovante de endereço (residencial e comercial, se diferentes)
    \item Número do recibo da última declaração de Imposto de Renda (se houver)
\end{itemize}

\subsection{Passo a passo}
\begin{enumerate}
    \item Acesse o Portal do Empreendedor: \url{https://www.gov.br/empresas-e-negocios}
    \item Clique em "Quero ser MEI"
    \item Preencha seus dados pessoais
    \item Escolha sua atividade econômica principal
    \item Informe o endereço onde será exercida a atividade
    \item Declare que está ciente das condições
    \item Finalize o cadastro
    \item Imprima o Certificado da Condição de Microempreendedor Individual (CCMEI)
\end{enumerate}

\section{Atividades permitidas para MEI}

Existem mais de 400 atividades permitidas, incluindo:

\begin{longtable}{|p{6cm}|p{8cm}|}
\hline
\textbf{Categoria} & \textbf{Exemplos de Atividades} \\
\hline
Comércio & Vendedor ambulante, lojista, vendedor de cosméticos \\
\hline
Indústria & Confecção de roupas, fabricação de produtos alimentícios \\
\hline
Serviços & Cabeleireiro, manicure, eletricista, encanador, fotógrafo \\
\hline
Alimentação & Lanchonete, restaurante, sorveteria, doceira \\
\hline
Tecnologia & Programador, designer gráfico, assistência técnica \\
\hline
Transporte & Motorista de táxi, mototaxista, entregador \\
\hline
\end{longtable}

\section{Obrigações do MEI}

\subsection{Mensais}
\begin{itemize}
    \item Pagamento do DAS (Documento de Arrecadação do Simples Nacional)
    \item Valores em 2024:
    \begin{itemize}
        \item Comércio/Indústria: R\$ 67,00
        \item Serviços: R\$ 71,00
        \item Comércio e Serviços: R\$ 72,00
    \end{itemize}
\end{itemize}

\subsection{Anuais}
\begin{itemize}
    \item Declaração Anual Simplificada (DASN-SIMEI)
    \item Prazo: até 31 de maio do ano seguinte
    \item Relatório mensal de receitas brutas
\end{itemize}

\section{Emissão de Nota Fiscal}

\subsection{Quando é obrigatória}
\begin{itemize}
    \item Venda para pessoa jurídica (empresa)
    \item Venda para consumidor final quando solicitada
    \item Prestação de serviços para pessoa jurídica
\end{itemize}

\subsection{Como emitir}
\begin{itemize}
    \item Nota Fiscal Eletrônica (NFe) - para comércio/indústria
    \item Nota Fiscal de Serviço Eletrônica (NFSe) - para serviços
    \item Utilize sistemas gratuitos disponibilizados pelos estados/municípios
\end{itemize}

\section{Controle financeiro}

\subsection{Relatório mensal obrigatório}
Deve conter:
\begin{itemize}
    \item Receita bruta mensal
    \item Nota fiscal emitida (se houver)
    \item Nota fiscal de compra (se houver)
\end{itemize}

\subsection{Dicas de organização}
\begin{itemize}
    \item Mantenha um caderno ou planilha de controle
    \item Separe conta pessoal da empresarial
    \item Guarde todos os comprovantes
    \item Anote todas as receitas e despesas
    \item Faça backup dos dados regularmente
\end{itemize}

\section{Desenquadramento do MEI}

\subsection{Situações que levam ao desenquadramento}
\begin{itemize}
    \item Faturamento superior a R\$ 97.200,00 no ano
    \item Faturamento superior a R\$ 81.000,00 por 2 anos consecutivos
    \item Contratação de mais de 1 empregado
    \item Participação em outra empresa
    \item Exercício de atividade não permitida
\end{itemize}

\subsection{Como proceder}
\begin{itemize}
    \item Comunique o desenquadramento no Portal do Simples Nacional
    \item Migre para Microempresa (ME) ou outra modalidade
    \item Ajuste as obrigações fiscais e tributárias
    \item Procure auxílio profissional se necessário
\end{itemize}

\section{Benefícios previdenciários}

\subsection{Contribuindo em dia, o MEI tem direito a:}
\begin{itemize}
    \item Aposentadoria por idade (65 anos homem, 62 anos mulher)
    \item Aposentadoria por invalidez
    \item Auxílio-doença
    \item Salário-maternidade
    \item Pensão por morte (para família)
    \item Auxílio-reclusão (para família)
\end{itemize}

\section{Dicas importantes}

\begin{itemize}
    \item Mantenha seus dados sempre atualizados
    \item Pague o DAS sempre em dia para evitar multas
    \item Faça a declaração anual dentro do prazo
    \item Mantenha controle rigoroso das receitas
    \item Consulte regularmente sua situação no Portal do Simples Nacional
    \item Procure capacitação em gestão empresarial
\end{itemize}

\section{Suporte e auxílio}

\subsection{Onde buscar ajuda}
\begin{itemize}
    \item Portal do Empreendedor: \url{https://www.gov.br/empresas-e-negocios}
    \item SEBRAE: Cursos e consultorias gratuitas
    \item NAF: Orientação fiscal gratuita
    \item Contador: Para casos mais complexos
\end{itemize}

\section{Contato NAF}

Para mais informações ou dúvidas sobre MEI, procure o Núcleo de Apoio Fiscal:

\begin{itemize}
    \item \textbf{Endereço:} Estácio Florianópolis
    \item \textbf{Telefone:} (48) 98461-4449
    \item \textbf{E-mail:} naf@estacio.br
    \item \textbf{Atendimento:} Segunda a sexta, das 8h às 18h
\end{itemize}

\vfill
\begin{center}
\footnotesize
Este guia foi elaborado pelo NAF - Núcleo de Apoio Fiscal da Estácio Florianópolis\\
Última atualização: \today
\end{center}

\end{document}