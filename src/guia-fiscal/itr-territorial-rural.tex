\documentclass[12pt,a4paper]{article}
\usepackage[utf8]{inputenc}
\usepackage[portuguese]{babel}
\usepackage{geometry}
\usepackage{fancyhdr}
\usepackage{graphicx}
\usepackage{xcolor}
\usepackage{hyperref}
\usepackage{enumitem}
\usepackage{titlesec}

\geometry{margin=2.5cm}

\pagestyle{fancy}
\fancyhf{}
\fancyhead[L]{NAF - Núcleo de Apoio Fiscal}
\fancyhead[R]{Estácio Florianópolis}
\fancyfoot[C]{\thepage}

\titleformat{\section}{\large\bfseries\color{blue!60!black}}{\thesection}{1em}{}
\titleformat{\subsection}{\normalsize\bfseries\color{blue!40!black}}{\thesubsection}{1em}{}

\begin{document}

\begin{center}
{\LARGE \textbf{GUIA COMPLETO}}\\[0.5cm]
{\Large \textbf{ITR - IMPOSTO TERRITORIAL RURAL}}\\[0.5cm]
{\large Núcleo de Apoio Fiscal - NAF}\\
{\large Estácio Florianópolis}\\[1cm]
\end{center}

\section{O que é o ITR?}

O Imposto sobre a Propriedade Territorial Rural (ITR) é um tributo federal que incide anualmente sobre a propriedade, o domínio útil ou a posse de imóvel localizado fora da zona urbana do município.

\section{Objetivo do ITR}

\begin{itemize}
    \item Estimular o melhor aproveitamento da terra
    \item Desestimular a manutenção de propriedades improdutivas
    \item Incentivar a função social da propriedade rural
    \item Arrecadar recursos para a União
\end{itemize}

\section{Quem deve declarar o ITR?}

\subsection{Estão obrigados}
\begin{itemize}
    \item Proprietários de imóveis rurais
    \item Titulares do domínio útil
    \item Possuidores a qualquer título
    \item Responsáveis pelos imóveis rurais em 1º de janeiro do ano-base
\end{itemize}

\subsection{Independente de}
\begin{itemize}
    \item Valor da propriedade
    \item Área da propriedade
    \item Grau de utilização
    \item Localização geográfica
    \item Resultado econômico da exploração
\end{itemize}

\section{Prazo para declaração}

\begin{itemize}
    \item \textbf{Prazo normal:} Até 30 de setembro
    \item \textbf{Declaração em atraso:} Após 30 de setembro (com multa)
    \item \textbf{Ano-base:} Situação em 1º de janeiro
\end{itemize}

\section{Como calcular o ITR}

\subsection{Fórmula básica}
ITR = VTN × Alíquota

Onde:
\begin{itemize}
    \item \textbf{VTN} = Valor da Terra Nua
    \item \textbf{Alíquota} = Depende da área e grau de utilização
\end{itemize}

\subsection{Valor da Terra Nua (VTN)}
\begin{itemize}
    \item Valor do imóvel excluindo construções, instalações e melhoramentos
    \item Base para cálculo do imposto
    \item Deve refletir o valor de mercado em 1º de janeiro
    \item Atualizado anualmente pelo contribuinte
\end{itemize}

\subsection{Grau de Utilização (GU)}
GU = (Área Utilizada ÷ Área Utilizável) × 100

\begin{itemize}
    \item \textbf{Área Utilizada:} Efetivamente explorada
    \item \textbf{Área Utilizável:} Própria para exploração agrícola/pecuária/florestal
    \item \textbf{Não inclui:} Área de preservação permanente, reserva legal, benfeitorias
\end{itemize}

\section{Tabela de alíquotas}

As alíquotas variam conforme a área total e grau de utilização:

\subsection{Área até 50 hectares}
\begin{itemize}
    \item GU maior que 80\%: 0,03\%
    \item GU de 65\% a 80\%: 0,20\%
    \item GU de 50\% a 65\%: 0,40\%
    \item GU de 30\% a 50\%: 0,70\%
    \item GU menor que 30\%: 1,00\%
\end{itemize}

\subsection{Área de 50 a 200 hectares}
\begin{itemize}
    \item GU maior que 80\%: 0,07\%
    \item GU de 65\% a 80\%: 0,40\%
    \item GU de 50\% a 65\%: 0,80\%
    \item GU de 30\% a 50\%: 1,40\%
    \item GU menor que 30\%: 2,00\%
\end{itemize}

\subsection{Área de 200 a 500 hectares}
\begin{itemize}
    \item GU maior que 80\%: 0,10\%
    \item GU de 65\% a 80\%: 0,60\%
    \item GU de 50\% a 65\%: 1,30\%
    \item GU de 30\% a 50\%: 2,30\%
    \item GU menor que 30\%: 3,30\%
\end{itemize}

\subsection{Área de 500 a 1.000 hectares}
\begin{itemize}
    \item GU maior que 80\%: 0,15\%
    \item GU de 65\% a 80\%: 0,85\%
    \item GU de 50\% a 65\%: 1,90\%
    \item GU de 30\% a 50\%: 3,30\%
    \item GU menor que 30\%: 4,70\%
\end{itemize}

\subsection{Área acima de 1.000 hectares}
\begin{itemize}
    \item GU maior que 80\%: 0,20\%
    \item GU de 65\% a 80\%: 1,00\%
    \item GU de 50\% a 65\%: 2,40\%
    \item GU de 30\% a 50\%: 4,20\%
    \item GU menor que 30\%: 6,00\%
\end{itemize}

\section{Isenções e imunidades}

\subsection{Imunidade}
\begin{itemize}
    \item Propriedades da União, Estados, DF e Municípios
    \item Propriedades de partidos políticos
    \item Propriedades de entidades sindicais dos trabalhadores
    \item Propriedades de instituições de educação e assistência social
\end{itemize}

\subsection{Isenção}
\begin{itemize}
    \item Pequenas glebas rurais (área e valor conforme lei)
    \item Áreas de preservação permanente
    \item Áreas de reserva legal
    \item Áreas de interesse ecológico
    \item Terras indígenas
\end{itemize}

\section{Documentos necessários}

\begin{itemize}
    \item CPF ou CNPJ do proprietário
    \item Escritura ou documento que comprove a propriedade
    \item Certidão de registro no INCRA (CCIR) atual
    \item Cadastro Ambiental Rural (CAR)
    \item Planta ou croqui do imóvel
    \item Relatório de atividades desenvolvidas
    \item Comprovantes de receitas e despesas (se exploração comercial)
\end{itemize}

\section{Programa gerador ITR}

\subsection{Como utilizar}
\begin{enumerate}
    \item Baixe o programa no site da Receita Federal
    \item Instale em seu computador
    \item Execute o programa
    \item Preencha os dados do imóvel
    \item Calcule o grau de utilização
    \item Informe o valor da terra nua
    \item Gere a declaração
    \item Transmita pela internet
\end{enumerate}

\subsection{Principais fichas}
\begin{itemize}
    \item \textbf{Identificação:} Dados do contribuinte e imóvel
    \item \textbf{Características:} Área total, utilizável, utilizada
    \item \textbf{Valor:} Valor da terra nua e benfeitorias
    \item \textbf{Exploração:} Atividades desenvolvidas
    \item \textbf{Resumo:} Cálculo do imposto devido
\end{itemize}

\section{Pagamento do ITR}

\subsection{Formas de pagamento}
\begin{itemize}
    \item Quota única: Até 30 de setembro (sem juros)
    \item Parcelado: Até 4 quotas (com juros de 1\% ao mês)
    \item DARF: Documento de Arrecadação da Receita Federal
\end{itemize}

\subsection{Valores mínimos}
\begin{itemize}
    \item ITR anual: R\$ 10,00
    \item Cada parcela: R\$ 50,00
    \item Se inferior, pagar em quota única
\end{itemize}

\section{Obrigações acessórias}

\begin{itemize}
    \item Manter CCIR (Certificado de Cadastro de Imóvel Rural) atualizado
    \item Apresentar Cadastro Ambiental Rural (CAR) quando exigido
    \item Conservar documentação por 5 anos
    \item Informar alterações cadastrais no INCRA
    \item Comunicar transferência de propriedade
\end{itemize}

\section{Multas e penalidades}

\subsection{Declaração em atraso}
\begin{itemize}
    \item Multa mínima: R\$ 165,74
    \item Máxima: 20\% do ITR devido
    \item Cálculo: 1\% ao mês sobre o imposto
    \item Juros Selic
\end{itemize}

\subsection{Informações incorretas}
\begin{itemize}
    \item Multa de 20\% sobre o imposto devido
    \item Juros de mora (Selic)
    \item Podem gerar autuação fiscal
\end{itemize}

\section{Fiscalização}

\subsection{A Receita Federal pode}
\begin{itemize}
    \item Verificar as informações declaradas
    \item Solicitar documentos complementares
    \item Realizar vistorias no imóvel
    \item Cruzar dados com outros órgãos
    \item Aplicar multas por irregularidades
\end{itemize}

\subsection{Dicas para evitar problemas}
\begin{itemize}
    \item Declare informações verdadeiras
    \item Mantenha documentação organizada
    \item Avalie corretamente o valor da terra
    \item Calcule adequadamente o grau de utilização
    \item Consulte profissionais quando necessário
\end{itemize}

\section{Dicas importantes}

\begin{itemize}
    \item Faça backup da declaração
    \item Guarde recibo de entrega
    \item Acompanhe o processamento
    \item Mantenha CCIR atualizado
    \item Conserve documentos por 5 anos
    \item Declare mesmo que isento ou imune
    \item Procure orientação profissional em casos complexos
\end{itemize}

\section{Contato NAF}

Para auxílio com o ITR, procure o Núcleo de Apoio Fiscal:

\begin{itemize}
    \item \textbf{Endereço:} Estácio Florianópolis
    \item \textbf{Telefone:} (48) 98461-4449
    \item \textbf{E-mail:} naf@estacio.br
    \item \textbf{Atendimento:} Segunda a sexta, das 8h às 18h
\end{itemize}

\vfill
\begin{center}
\footnotesize
Este guia foi elaborado pelo NAF - Núcleo de Apoio Fiscal da Estácio Florianópolis\\
Última atualização: \today
\end{center}

\end{document}